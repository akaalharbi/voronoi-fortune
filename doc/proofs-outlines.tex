\documentclass{article}
\usepackage[english]{babel}
\usepackage[utf8]{inputenc}
\usepackage{fancyhdr}
\usepackage{amsmath,amsthm,amssymb,amsfonts}

\newcommand{\N}{\mathbb{N}}
\newcommand{\Q}{\mathbb{Q}}

% Fancy header 
\pagestyle{fancy}
\fancyhf{}
\rhead{}
\lhead{Proofs outlines}
\rfoot{Page \thepage}

%%%%%%%%%%%%%%%%
\begin{document}

\section{Goal}
\section{Definitions}
\subsection{f}
\subsection{Cells}
\subsection{complete}
\subsection{incomplete}

\section{Functions}
\subsection{Parabola Functions}
The main method is to show that a solution belongs to two curves which boils down to algebraic manipulations.

\subsection{Lines, Circles, and distances}
\section{Events}
\subsection{site event}
\subsection{circle event}
\end{document}